\documentclass[12pt]{article}
\usepackage{amsmath,amsthm,amsfonts, latexsym, amssymb}
\usepackage{graphicx}

\theoremstyle{definition}
\newtheorem{prob}{Problem}
\newtheorem{theorem}{Theorem}

\def\rr{\mathbb{R}}
\def\cc{\mathbb{C}}
\def\zz{\mathbb{Z}}
\def\qq{\mathbb{Q}}
\def\nn{\mathbb{N}}
\def\aa{\mathbb{A}}
\def\pp{\mathcal{P}}
\def\ff{\mathfrak{F}}

\setlength{\textheight}{9in} \setlength{\textwidth}{6.0in}
\setlength{\topmargin}{0.0in} \setlength{\headheight}{0.0in}
\setlength{\headsep}{0.0in} \setlength{\leftmargin}{0.0in}
\setlength{\oddsidemargin}{0.0in} \setlength{\parindent}{1pc}
\textwidth 6.5truein
\textheight 9truein

\newcounter{probparts}
\newenvironment{parts}
    {\begin{list}
        {(\alph{probparts})}
        {\setlength{\itemindent}{0cm}
        \setlength{\leftmargin}{.9cm}
        \setlength{\labelsep}{.2cm}
        \setlength{\labelwidth}{.7cm}
        \setlength{\listparindent}{0cm}
        \usecounter{probparts}}}
    {\end{list}}

\newenvironment{tightcenter}{%
  \setlength\topsep{0pt}
  \setlength\parskip{0pt}
  \begin{center}
}{%
  \end{center}
}

\begin{document}
\pagestyle{empty}

\flushleft


\textbf{Tung Phan \hfill MATH291 - F Session \hfill 4/23/2014}

\vspace{4ex}

\begin{theorem}
Suppose $(P, w)$ contains $(Q, w')$ as a convex subposet. If $(Q, w')$ has multiplicity, then so does $(P,w)$.
\end{theorem}

\begin{proof}
Let $L_{(Q,w')}$ be the set of all linear extensions of $(Q,w')$.\\
Since $(Q,w')$ has multiplicity, there exist $l_1, l_2 \in L_{(Q,w')}$ such that they have the same $F$ function. \\
\textit{Case 1:} Every labeled node in $(P,w)$ belongs in $(Q,w')$ and thus, $(P,w) = (Q,w')$. Hence, $l_1, l_2 \in L_{(P,w)}$ and $(P,w)$ has multiplicity.\\
\textit{Case 2:} There exists labeled node(s) in $(P,w)$ that don't belong in $(Q,w')$. 

\end{proof}
\end{document}
